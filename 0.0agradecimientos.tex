\chapter*{Agradecimientos}

Quiero expresar mi profunda gratitud a todas las personas que contribuyeron a la realización de este trabajo.

En primer lugar, agradezco al Dr. José Ulises Márquez Urbina por su orientación experta y consejos pacientes que fueron fundamentales durante todo el proceso de esta investigación. Agradezco también al Dr. Israel Emmanuel Ambriz Lobato por su unvaluable ayuda en la construcción y la implementación del modelo, así como a la Dra. Graciela María de los Dolores González Farías por su guía experta en el ámbito estadístico.

Quiero reconocer al Centro de Investigaciones en Matemáticas (CIMAT), donde encontré a profesores excepcionales que no solo me inspiraron y fomentaron mi pasión por la ciencia, sino que también me alentaron constantemente a seguir adelante.

A mis padres, Gabriela Silva y Guillermo Quintana, les debo un agradecimiento especial por su inquebrantable apoyo y amor desde mi infancia. Su compromiso y dedicación han sido una fuente inagotable de inspiración para mí. A mi hermano Guillermo A. Quintana, le agradezco sus consejos y orientación, tanto en este trabajo como en la vida en general.

No puedo pasar por alto a mis fieles compañeras, Nyx y Kora, cuya presencia constante y afectuosa durante las largas horas de escritura de esta tesis fueron un bálsamo en momentos de frustración.

A mis queridos amigos, Sara, Zaira, Esaul, Franklin y Alex, les estoy profundamente agradecido por ser un apoyo emocional fundamental a lo largo de mi trayecto de maestría. su compañía, risas y ánimos me han impulsado a dar lo mejor de mí en todo momento.

Este logro no es solo mío, sino también suyo. Agradezco de corazón por estar a mi lado en este largo y gratificante viaje.

Adicionalmente, le doy mi agradecimiento al Consejo Nacional de Humanidades, Ciencias y Tecnologías (CONAHCyT) por el apoyo económico necesario durante mis estudios de maestría en Guanajuato.
