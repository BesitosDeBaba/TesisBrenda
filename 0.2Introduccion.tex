\pagenumbering{arabic}
\prefacesection{Introducción}
\markboth{Introducción}{}


La diabetes mellitus es una enfermedad metabólica crónica que representa un desafío de salud pública a nivel mundial. Está caracterizada por mantener los niveles altos de glucosa en sangre, lo cual tiene que ver con la producción de insulina de nuestro cuerpo.A pesar de los avances en el diagnóstico y tratamiento, la prevalencia de la diabetes sigue aumentando, y su impacto en la calidad de vida de los individuos y en los sistemas de salud es significativo. Una de las dificultades en el manejo efectivo de la diabetes radica en la falta de modelos precisos y adaptados a poblaciones específicas, lo que puede llevar a diagnósticos erróneos o a la subestimación del riesgo de desarrollar la enfermedad \cite{HOMAMex}.

De acuerdo con las estadísticas tomadas de la página del Instituto Nacional de Estadística y Geografía (INEGI) del período enero-junio del 2023, en México la diabetes es la segunda causa de muerte \cite{INEGI}. Año con año la tasa de personas con esta enfermedad aumentó súbitamente, lo alarmante de esta situación es que muchas personas en México la padecen y no lo saben. Las causas de esta enfermedad son muy variadas, por ejemplo, malos hábitos alimenticios, línea genética, llevar una vida sedentaria, entre otras \cite{PromoSaludMexico}.

En este contexto, el presente trabajo se enfoca en abordar esta brecha identificando y proponiendo un modelo ajustado específicamente para la población mexicana. Aún cuando existen índices de medición para la resistencia a la insulina, como el índice HOMA o el índice de Matsuda, estos se encuentran calibrados principalmente para poblaciones caucásicas o para individuos en su mayoría hombres. La falta de análisis previos para poblaciones particulares, así como la imposibilidad de incluir individuos representativos de todas las diversidades existentes, resalta la necesidad de desarrollar un modelo específico que considere las características genéticas, ambientales y culturales únicas de la población mexicana.

El objetivo principal de este estudio es proponer un modelo de predicción de la diabetes mellitus tipo 2 que considere las particularidades específicas de la población mexicana, utilizando datos biométricos y clínicos disponibles para cualquier tipo de comunidad. Se espera que este modelo contribuya a mejorar la precisión del diagnóstico, identificar de manera temprana a las personas en riesgo de desarrollar la enfermedad y, en última instancia, facilitar la implementación de intervenciones preventivas y estrategias de manejo personalizadas.

Afortunadamente, en la comunidad mexicana hay varios investigadores comprometidos en encontrar soluciones para esta enfermedad tan prevalente. La Dra. Adriana Monroy Guzmán, en colaboración con el Centro de Investigación en Matemáticas (CIMAT), ha dedicado mucho esfuerzo y dedicación a la recopilación de una base de datos que incluye mediciones de glucosa e insulina durante la prueba oral de tolerancia a la insulina, así como datos adicionales como el índice de masa corporal (IMC), la presión arterial, los porcentajes de grasa, entre otros. Con esta invaluable información y el continuo trabajo del destacado estudiante de doctorado Israel Emmanuel Ambriz Lobato, quien ha investigado ampliamente sobre el uso de cópulas y su implementación en modelos estadísticos, y ha desarrollado un modelo de regresión. Esta investigación ha sido realizada bajo la tutela de la Dra. Graciela González Farías y el Dr. José Ulises Márquez Urbina, con la supervisión médica de la Dra. Monroy.

Cabe destacar que una de las principales motivaciones es lograr una mejor comprensión de las relaciones modeladas en los métodos de Machine Learning y evitar el uso de `cajas negras', es fundamental realizar un análisis estadístico detallado. Este análisis permite verificar si el ajuste del modelo de regresión tiene sentido y está en línea con los conocimientos médicos existentes. Al integrar métodos estadísticos rigurosos, se pueden interpretar los resultados de manera más transparente y confiable, asegurando que las predicciones y conclusiones obtenidas sean válidas y útiles desde una perspectiva clínica.

En el Capítulo \ref{insulina}, se presenta un análisis detallado del proceso de la homeostasis de la glucosa con el objetivo de comprender los factores clave que regulan los niveles de glucosa en la sangre. Además, se aborda el síndrome metabólico, resaltando el proceso de fallo en la síntesis de la insulina y destacando algunas características importantes para identificar este trastorno.

Se describe la prueba oral de tolerancia a la glucosa como una medida para la efectiva de la sensibilidad a la insulina, así como el test de Clamp, considerado el estándar por excelencia para medir la resistencia a la insulina. Sin embargo, debido a su complejidad y carácter invasivo, el test de Clamp rara vez se utiliza en la práctica clínica. Por lo tanto, la comunidad científica ha propuesto varios índices como referencia para evaluar la resistencia a la insulina.

Además, se mencionan diversas aproximaciones que la comunidad científica ha empleado para abordar esta problemática, incluyendo modelado con ecuaciones diferenciales, modelos bayesianos, entre otros enfoques.

En el Capítulo \ref{RegCuanDatosFun}, se introduce el modelo matemático con el que se trabajará, comenzando con la descripción de conceptos fundamentales del área de probabilidad. Se exploran conceptos como el cuantil, las funciones de distribución multivariadas y las funciones de distribución marginales, entre otros, haciendo especial énfasis en su formulación muestral. El objetivo principal de esta sección es sentar las bases para comprender la regresión cuantil, un enfoque estadístico que difiere significativamente de la clásica regresión lineal convencional.

La regresión cuantil presenta varias ventajas, siendo una de las más destacables su robustez ante la presencia de valores atípicos. Además, esta técnica permite realizar la regresión con un cierto nivel de confianza, lo cual facilita la estimación de intervalos de confianza con una determinada fiabilidad. Para ilustrar estas diferencias y ventajas, se presenta un ejemplo que contrasta la regresión cuantil con la regresión lineal tradicional, resaltando cómo la primera puede proporcionar resultados más robustos y útiles en ciertos escenarios.


Como última herramienta fundamental, se introduce el concepto de datos funcionales, que consisten en representar las observaciones como funciones, cuya característica distintiva es que no solo se considera el valor de la variable, sino también su forma y escala. Este enfoque resulta de vital importancia para el presente trabajo, dado que el objetivo principal es modelar la relación entre los niveles de glucosa y de insulina en los pacientes utilizando las mediciones obtenidas durante la prueba de tolerancia a la glucosa oral.

Para lograr esto, se emplea la técnica de las bandas de profundidad, la cual permite crear una variable respuesta escalar que conserva su esencia funcional. Esta variable respuesta puede ser modelada utilizando copulas, una herramienta poderosa en el análisis de dependencia entre variables aleatorias. 

El Capítulo \ref{CapCopulas} se centra en la introducción del concepto de copula y su relación con la función de distribución, explorando además su estimación muestral y algunas técnicas exploratorias para comprender su relación y su modelado paramétrico. El objetivo principal de esta sección es preparar al lector para la presentación de las R-vines, una forma particular de descomponer la distribución en copulas a pares.

Las R-vines tienen la característica de poder ser visualizadas como grafos, lo que facilita su comprensión e interpretación. Aunque existen varias formas de realizar esta descomposición, se optará por la estructura de descomposición D-vine como enfoque central debido a su clara interpretación en la unión de nodos y su estructura de grafo.

El objetivo final de este capítulo es unir el concepto de cópula como modelo con la regresión cuantil como técnica de estimación. De esta manera, una vez explorada la parte matemática, se podrá describir el algoritmo computacional y las herramientas que se utilizarán para construir el modelo de regresión, además de presentar uno de los principales resultados de la tesis: la implementación de la paquetería en R, \textbf{deerVineReg}.


El Capítulo \ref{Resultados} se puede dividir en 2 partes. La primera comienza con un análisis exploratorio de la base de datos para tener una noción de cómo está conformada, haciendo énfasis en las estructuras de las sub-bases determinadas por los diferentes prediagnósticos, y así poder comprender la relación entre insulina y glucosa. Posteriormente, se detalla la construcción e interpretación de los gráficos implementados para la visualización de dependencias en cópulas a pares, cuya idea fue tomada del artículo \textit{Copula-based statistical dependence visualizations}. 

Adicionalmente, se presenta la gráfica de efectos, cuya utilidad es observar cómo varía el estimador cuantil en respuesta a cambios en un predictor específico. Esta herramienta permite identificar la relación y el impacto de cada predictor en el resultado de interés, facilitando la interpretación de los modelos y la toma de decisiones. Finalmente, se culmina con la presentación de la librería \textit{deerVineReg}, destacando sus capacidades y beneficios en la implementación de modelos de cópulas para la predicción de la relación entre insulina y glucosa.

En la segunda parte, se aborda el modelo implementado con toda la población, que consta de cuatro covariables predictoras. Aunque si se desea ver los modelos construidos solo para hombres o mujeres, estos se pueden encontrar en el Apéndice \ref{ApendiceA}. Se discuten las dependencias observadas con los gráficos presentados anteriormente y se evalúa el ajuste del modelo.

A partir de este análisis, se concluye que un modelo con tres variables es el que mejor extrae las características significativas, optimizando así la precisión y la interpretabilidad del modelo. Esto cumple con el objetivo de la tesis, que consiste en desarrollar un modelo específico para la población mexicana, proporcionando una herramienta más adaptada y eficaz para el diagnóstico y manejo de la diabetes en esta población.









