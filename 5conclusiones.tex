\chapter{Conclusiones}

La Diabetes Mellitus representa un desafío a nivel mundial, siendo la cuarta principal causa de muerte durante el 2023 y la segunda en México \cite{INEGI}. A pesar de que se han propuesto diversos modelos matemáticos para comprender esta enfermedad, una de las mayores dificultades en su manejo efectivo radica en la falta de modelos precisos y adaptados a poblaciones específicas. Este trabajo propone un modelo ajustado especialmente para la población mexicana, con el objetivo de abordar esta brecha y ofrecer una herramienta que pueda ser utilizada en cualquier tipo de comunidad.

El problema planteado se abordó primero desde la parte biológica, pues es esencial comprender el proceso de la homeostasis de la glucosa para extraer los factores clave que regulan los niveles de glucosa en la sangre y el origen de los síndromes metabólicos. Además, se hizo un breve recorrido sobre los modelos matemáticos anteriormente propuestos para esta enfermedad, evaluando sus fortalezas y limitaciones.

Con las curvas de información de niveles de glucosa e insulina de alrededor de 1500 individuos de la prueba oral de tolerancia a la glucosa, provenientes de la población mexicana proporcionadas por la Dra. Adriana Monroy, se propuso emplear la técnica de las bandas de profundidad junto con una medida para la efectiva de la sensibilidad a la insulina. Esta técnica y medida permite crear una variable respuesta escalar que conserva su esencia funcional, idea desarrollada por el estudiante de doctorado de CIMAT, Israel Emmanuel Ambriz Lobato, en su trabajo de tesis. Haciendo uso de regresión cuantil, esta técnica permite realizar la regresión con un cierto nivel de confianza, además tiene varias ventajas, siendo una de las más destacables su robustez ante la presencia de valores atípicos.

Como herramienta final, se propuso hacer el ajuste utilizando un modelo de cópulas llamado D-Vine. La idea fundamental es la descomposición de la función de distribución de un vector en cópulas bivariadas, las cuales se determinan con un grafo o árboles. Utilizando el grafo, es sencillo visualizar las dependencias condicionales o cópulas. Esto se implementó haciendo uso del paquete \textbf{VineCopula} y varias técnicas de visualización de medidas de dependencia para realizar análisis exploratorio.

Adicionalmente, se han computado gráficos de efectos para visualizar la variabilidad y la relación entre la variable respuesta y su impacto en la predicción de la diabetes. Estos gráficos han demostrado ser fundamentales para identificar patrones y tendencias, así como para resaltar variables que no aportan información significativa al modelo. Esta capacidad de visualización facilita la selección de variables relevantes y mejora la interpretabilidad y precisión de los resultados.

Después de ajustar los dos modelos de regresión con 3 y 4 variables, gracias a todas las herramientas de visualización implementadas y la significancia de los test de independencia, fue sencillo poder distinguir la importancia de las variables en el modelo y así decidir si desecharlas o no. Además, la visualización de las curvas de nivel de cada cópula fue otra herramienta útil, ya que permitió observar y comparar cómo son las interacciones de todas las variables hasta las condicionales. Estas curvas de nivel nos brindan una representación gráfica de la estructura de dependencia entre las variables, lo que facilita la identificación de patrones y relaciones complejas que pueden no ser evidentes al analizar los datos de forma individual.

Adicionalmente, se han computado gráficos de efectos para visualizar la variabilidad y la relación entre la variable respuesta y su impacto en la predicción de la diabetes. Estos gráficos han demostrado ser fundamentales para identificar patrones y tendencias, así como para resaltar variables que no aportan información significativa al modelo. Esta capacidad de visualización facilita la selección de variables relevantes y mejora la interpretabilidad y precisión de los resultados.

En conclusión, el modelo propuesto no solo busca mejorar la comprensión y el manejo de la diabetes en la población mexicana, sino también proporcionar una herramienta accesible y aplicable a comunidades con recursos limitados. Al ofrecer un modelo simplificado y visualmente intuitivo, se espera que profesionales de la salud puedan tomar decisiones más informadas y efectivas en la lucha contra esta enfermedad.

Como trabajo a futuro, se podría seguir experimentando con otras variables y culminar con alguna aplicación de fácil uso en los hospitales. Por otro lado, en la parte computacional, se planea continuar con el desarrollo de la paquetería que implementa el modelo D-vine y poder implementar más modalidades a las cópulas. Actualmente, el modelo construido solo utiliza cópulas paramétricas. Sin embargo, se busca permitir que el usuario pueda asignar una cópula paramétrica, no paramétrica, o una combinación específica de cópulas, ofreciendo así mayor flexibilidad y precisión en el ajuste del modelo. Además, se planea liberar esta paquetería en CRAN para que esté disponible al público en general, facilitando su uso y adopción por parte de la comunidad científica y médica.