\chapter*{Resumen}

La diabetes, una enfermedad crónica que altera el procesamiento de la glucosa, conlleva graves consecuencias, como enfermedades cardíacas, accidentes cerebrovasculares, daño nervioso, problemas visuales, renales y amputaciones. El creciente número de personas afectadas, particularmente en la población mexicana, es motivo de preocupación. Aunque existen índices establecidos para evaluar el riesgo de resistencia a la insulina, como el HOMA o Matsuda, su estimación hecha con poblaciones caucásicas o predominantemente masculinas limita su utilidad generalizada.

Este estudio presenta un enfoque innovador para estimar el riesgo de resistencia a la insulina, utilizando variables fácilmente accesibles para comunidades de bajos recursos y adaptado específicamente a la población mexicana. Se analizan datos de curvas de glucosa e insulina obtenidas durante la prueba de tolerancia a la glucosa y datos biométricos adicionales proporcionados por la Dra. Adrianna Monroy del hospital General de México. Se emplea un enfoque de análisis de datos funcionales, junto con el concepto de bandas de profundidad desarrollado por López Pintado y Romo \cite{boxplotFun}, y posteriormente extendido y mejorado por Emmanuel Ambriz en su tesis doctoral, para modelar las dependencias mediante copulas y realizar regresiones mediante técnicas de regresión cuantil.

Se exploran diversos modelos con diferentes conjuntos de variables y secciones de la población (hombres y mujeres), seguidos de una comparación de su desempeño. Este estudio promete proporcionar una herramienta más precisa y personalizada para la evaluación del riesgo de resistencia a la insulina, crucial para la prevención y manejo de la diabetes en la población mexicana. 

Además, como parte del análisis de resultados, se incluyen descripciones gráficas de las dependencias obtenidas con el modelo ajustado en cada nivel o árbol de la D-vine. Así como las tablas resumen, las cuales, utilizando las medianas de los subgrupos determinados por la glucosa basal y el índice de masa corporal, pueden proporcionar un prediagnóstico. Este enfoque gráfico facilita la visualización y comprensión de las interacciones entre variables, mejorando la interpretación de los resultados y la toma de decisiones clínicas.






